\documentclass[12pt]{article}
\usepackage{amsmath, amssymb,amscd}
\usepackage{xypic}
\usepackage{setspace}
\usepackage{enumerate}
\pagestyle{empty}
\setlength{\parindent}{0in}
\oddsidemargin 0in
\textwidth 6.25in
\topmargin 0in
\textheight 9in
\doublespace
\usepackage{amsthm}
\usepackage{enumitem}
\newenvironment{faq}{\begin{description}[style=nextline]}{\end{description}}



\begin{document} 

Surya Gaddipati\\
September 1, 2013\\
MATH 330 
\begin{faq}
\item[1a.Determine $gcd(2^4 \cdot  3^2 \cdot 5 \cdot 7^2, 2 \cdot 3^3 \cdot 7 \cdot 11)$]
  $gcd= 2 \cdot 3 \cdot 7 = 42$
\item[1b. Determine $lcm(2^3 \cdot  3^2 \cdot 5 , 2 \cdot 3^3 \cdot 7 \cdot 11)$]
  $lcm= 2^3 \cdot 3^3  = 8 \cdot 9 = 72$

\item[2.If $a$ and $b$ are postive integers then $ab = lcm(a,b)\cdot gcd(a,b)$]
  Let $d = gcd(a,b)$ , then
  \begin{equation}
    a\cdot b = p\cdot d \cdot q\cdot d
  \end{equation}
  We know that,
  $lcm(a,b) = k * lcm(x,y)$ where $k|a$ and $k|b$ and $x=a/k$ and $y=b/k$ \\
  We also know that, $lcm(a,b) = a\cdot b$ if $a$ and $b$ are relatively prime.\\
  Substituting this in equation 1, 
  \begin{align}
    a\cdot b = d \cdot d \cdot lcm(p,q)  \\
    a\cdot b = d \cdot lcm(a,b) \\ 
    a\cdot b = gcd(a,b) \cdot lcm(a,b) 
  \end{align}
\item[3. Euclidean Algorithm to find $gcd(34,126)$ write it as a linear combination of 34 and 126]
  \begin{align}
    126 = 3 \cdot 34 + 24 \\
    34 = 1 \cdot 24 + 10 \\
    24 = 2 \cdot 10 + 4 \\
    10 = 2 \cdot 4 + 2 \\
    4 = 2 \cdot 2 + 0\\
    gcd(34,126) = 2 
  \end{align}
  If, $d = gcd(a,b)$, then $d$ can be written as $d = ma + kb$. From above,
  \begin{align}
    24 = 126 - 3 \cdot 34 \\
    10 = 34 - 1 \cdot 24 \\
    4 = 24 - 2 \cdot 10 \\
    2 = 10 - 2 \cdot 4
  \end{align}
  Substituting back,
  \begin{align}
    2 = ( 34 -24 ) - 2(24 - 2 \cdot 10) \\
    2 = 34 - 126 + 3 \cdot 34  -2 ( 126 - 3 \cdot 34 - 2( 34 -24)) \\
    2  = 4 \cdot 34  - 126  - 2\cdot126 + 6\cdot34 + 4\cdot34 - 4\cdot24 \\
    2 = 14\cdot34 - 3\cdot126 - 4\cdot 126 + 12\cdot34 \\
    2 = 26\cdot34 - 7\cdot126
  \end{align}

\item[4. Let $p1,p2,..pm$ be primes. Show that p1p2...pn+1 is not divisible by any of those primes.]
  Let $P = p1p2...pn+1$ if $P$ was divisible by $p$ in that list the $p$ must divide $P - p1,p2,..pm=1$ but $p$ doesn't divide $1$. We have a contradiction.\\
  Hence no $p$ in that set divides $P$

\item[5. Prove that there are infinitely many primes.] \\
  \begin{proof}
    Lets assume that there are finite number of primes given by set $S = \left\{  {p1, p2,....,pn}\right\} $.\\
    Let $P = p1p2...pn+1$, as previously stated $P$ is  not divisible by any member of $S$, which means that $P$ is a new prime number which
    violates our intial assumption that there are only a finite set of prime numbers.
  \end{proof}
\end{faq}

\end{document}
