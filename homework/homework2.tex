\documentclass[12pt]{article}
\usepackage{amsmath, amssymb,amscd}
\usepackage{xypic}
\usepackage{setspace}
\usepackage{enumerate}
\pagestyle{empty}
\setlength{\parindent}{0in}
\oddsidemargin 0in
\textwidth 6.25in
\topmargin 0in
\textheight 9in
\doublespace
\usepackage{amsthm}
\usepackage{enumitem}
\newenvironment{faq}{\begin{description}[style=nextline]}{\end{description}}



\begin{document} 
Surya Gaddipati\\
\today \\
MATH 330 
\begin{faq}
\item[ Let S be the set of real numbers.If$a,b \in S$,define $a\sim b$ if $a - b$ is an integer. Show that $\sim$ is an equivalence relation on S. Describe the equivalence classes of S.]
 Verifing properties of a equivalance relation

 \begin{itemize}
   \item reflexive
     $(a,a) = a - a = 0$ and $0$ is an integer
   \item symmetric
     Assume $a \equiv b$, $b - a = -(a -b)$ if $ a -b$ is an integer then $-(a -b)$ is an integer too. Hence $(a,b)\in R$ implies $(b,a)\in R$
   \item tansitive
Assume $a \equiv b$ and  $b \equiv c$ which means $a-b$ and $b-c$ are integers  \\
$a - c = (a -b) + (b -c)$ . Sum of two integers is an iteger hence it implies $a\equiv c$.
 \end{itemize}

Equivalance Classes: Equivalance realation partions $S$ into set of Integers with 
\newline
\newline
\newline

\item[ Let S be the set of real numbers.If$a,b \in S$,define $aRb$ if $ab\geq0$. Is $R$ an equivalance relation on $S$?]
  No, because tansitive property doesn't hold. Let $a < 0$ and $b=0$ and $c > 0$  
$ab\geq0$ and $bc\geq0$ but $ac < 0$. It would have  been an equivalance realtion had it been $ab>0$.

\pagebreak
\item[Is $D_3$ albenian]
  No. From previous table $R_1K_1 = K_2$ but $K_1R_2=K_3$. So commutative property doesn't hold

\item[Find elements A,B ,and C in D4 such that $AB=BC$ but $A\neq C$. (Thus, “cross cancellation” is not valid.)]
  from the cayley's table for $D_4$ in the book, 
  $R_{90} V = D'$ \\ 
  $VR_{270} = D'$ 
  but  $R_{90} \neq R_{270}$
\end{faq}
\end{document}
